\section{Attori}
Come prima cosa \`e necessario definire con chiarezza quali sono i vari
``attori'' che formano il sistema della soluzione.
Si proceder\`a dunque con una panoramica generale su di essi.

\subsection{ACME}
L'attore principale della soluzione \`e l'azienda ACME: data per\`o la
dimensione di tale entit\`a si \`e ritenuto utile suddividerla in
diversi attori, che corrispondono ai diversi dipartimenti dell'azienda.

\subsubsection{Ufficio}
L'ufficio \`e l'attore che modella il processo principale dell'azienda
ACME, occupandosi dell'interazione con il cliente, con il magazzino
primario (e dunque con i magazzini secondari) e con l'istituto di
credito. Le modalit\`a di interazione con gli altri attori verr\`a
descritta con precisione nella sezione relativa alla progettazione dello
schema BPMN.

\subsubsection{Gestione magazzini}
I magazzini possono essere di due categorie: primario e secondario.
I magazzini vengono identificati tramite un {\tt id} numerico univoco,
assegnato nella seguente modalit\`a: il magazzino primario ha
{\tt id = 0}, e i magazzini secondari vengono numerati progressivamente
a partire da questo. In questo progetto sono stati implementati un
magazzino primario (dunque, {\tt id = 0}) e due magazzini secondari
({\tt id = 1, 2}).

Il magazzino primario funge da coordinatore dei magazzini secondari,
ponendosi dunque come intermediario tra l'ufficio ed i magazzini
secondari. \linebreak
Il magazzino primario si occupa inoltre di tutte le funzioni
dell'officina e del reparto che si occupa della preparazione delle
spedizioni.
Il magazzino primario comunica inoltre con la ditta di spedizione e con
il fornitore delle componenti.

\subsubsection{Officina}
Come detto in precedenza, l'officina \`e contenuta all'interno della
stessa struttura del magazzino primario. \`E la sede delle operazioni di
assemblaggio delle componenti e dei cicli. All'interno della nostra
soluzione, \`e un attore ``fittizio'', poich\'e le sue funzionalit\`a
vengono completamente implementate all'interno delle operazioni del
magazzino primario.

\subsection{Cliente}
Il cliente \`e l'attore che inizia il processo principale dell'azienda
ACME.
Come l'officina, non gode di un'implementazione effettiva, ma viene
simulato tramite richieste e risposte nelle interazioni con l'ufficio.
Un cliente pu\`o essere di due tipologie diverse: un privato o un
rivenditore. Tale distinzione non \`e per\`o rilevante dal punto di
vista dell'implementazione.

\subsection{Fornitore}
Il fornitore \`e l'attore che si occupa della fornitura delle componenti
non presenti nei magazzini. Se durante la ricerca di una componente nei
magazzini essa non viene trovata, viene mandata una richiesta al
fornitore, che si occuper\`a di spedirlo al magazzino principale. Come
scelta, si consideri che il fornitore avr\`a sempre le componenti
richieste dal magazzino disponibili. La gestione di altre operazioni non
viene considerata.

\subsection{Istituto di credito}
L'istituto di credito \`e un attore che deve simulare la verifica
dell'avvenuto pagamento da parte del cliente.
Le interazioni avvengono solamente con l'ufficio.
L'implementazione simula in maniera semplice la generazione di un esito,
tramite un sistema implementato nel seguente modo: viene selezionato un
numero casuale, e se esso \`e minore di {\tt 0.3}, l'istituto
restituisce {\tt false}; altrimenti restituisce {\tt true}. Si \`e
preferito ridurre le possibilit\`a che uscisse un esito negativo per
rendere pi\`u verosimile la soluzione. Non sono previste altre
operazioni.

\subsection{Ufficio legale}
L'ufficio legale \`e un attore fittizio, senza implementazione. Il suo
unico ruolo \`e quello di ricevere le notifiche di mancato pagamento da
parte dell'ufficio. Non sono previste operazioni che coinvolgano
ulteriormente tale attore.
