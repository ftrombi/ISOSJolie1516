\section*{Introduzione}
Il progetto consiste nell'implementazione di una \textit{Service
Oriented Architecture} (SOA) che rispecchi le necessit\`a di un'azienda
fittizia ma con processi verosimili.

L'introduzione di tale relazione descrive brevemente il dominio del
problema e i vincoli imposti alla soluzione.

Il primo capitolo si concentra sulla presentazione degli attori
coinvolti nei processi implementati.

Il secondo capitolo riguarda la descrizione dello schema BPMN utilizzato
per la progettazione dei processi che compongono la soluzione.

Il terzo capitolo mostra i dettagli implementativi della soluzione,
soffermandosi dunque su aspetti pi\`u tecnici.

\subsection*{Dominio del problema}
L'azienda presa come esempio \`e la \textit{ACME}, che si occupa di
fornire cicli assemblati oppure componenti atomiche.
La ACME si basa su un sistema di magazzini, cos\`i organizzato: esiste
un magazzino primario, adibito alla coordinazione di un insieme di
magazzini secondari. Il magazzino primario contiene anche l'officina,
per l'assemblaggio dei cicli.

La ACME riceve dunque ordini dai clienti, con i quali interagisce
durante diverse fasi: l'accertamento delle compatibilit\`a delle
eventuali customizzazioni, l'accettazione del preventivo, il pagamento
dell'acconto e del saldo. Gli ordini contengono una serie di
particolari, quali modello base, la colorazione e le eventuali
customizzazioni per gruppo frenante, gruppo sterzo, trasmissione e
ammortizzatori.

L'azienda deve interagire con una ditta di spedizioni per consegnare gli
oggetti ai clienti.
Normalmente le spedizioni partono dal magazzino primario; ma nel caso
l'ordine contenga solamente componenti che non richiedono assemblaggio,
\`e possibile spedire direttamente dal magazzino pi\`u vicino al
cliente, separatamente.
Quando si procede all'assemblaggio, si riservano le componenti
necessarie e le si spostano, se necessario, nel magazzino primario;
se non presenti vengono ordinate presso un fornitore esterno, unico.
Il costo dell'ordine viene calcolato tramite costo delle componenti e
costo di spedizione.

I pagamenti che il cliente effettua verso ACME vengono verificati
tramite un istituto di credito.
Diversi particolari non ancora indicati verranno mostrati in seguito
insieme all'implementazione adottata.

\subsection*{Vincoli}
I vincoli imposti alla soluzione sono stati i seguenti:
\begin{itemize}
  \item implementazione della gestione dei magazzini tramite SOA di
  servizi Jolie;
  \item esternalizzazione di alcune capability, quali il calcolo delle
  distanze \linebreak geografiche, l'istituto di credito e il fornitore delle
  componenti;
  \item utilizzo di \textit{API REST} per l'implementazione dei servizi
  di calcolo delle distanze geografiche e dell'istituto bancario;
  \item utilizzo di logica elementare per l'implementazione dei servizi,
  in modo da simulare comportamenti verosimili.
\end{itemize}

\subsection*{Descrizione della soluzione}
L'implementazione della soluzione \`e stata affrontata basandosi sulle
tre componenti principali richieste. Si sono dunque progettati ed
implementati:
\begin{itemize}
  \item lo schema BPMN del processo principale dell'azienda ACME,
  utilizzando il tool di progettazione grafica \textit{Signavio};
  \item il processo principale dell'ACME utilizzando un sistema BPMS,
  \textit{Camunda};
  \item la gestione dei magazzini, tramite una SOA di servizi Jolie;
  \item le capability esterne, sotto forma di servizi, utilizzando
  Jolie.
\end{itemize}
