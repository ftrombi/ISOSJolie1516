\section{Diagramma BPMN}
Il diagramma BPMN \`e stato realizzato utilizzando il software di
modellazione online \textit{Signavio}.
Sono state modellate diverse \textit{pool}: la principale, in chiaro,
\`e quella dell'azienda ACME, suddivisa al proprio interno in tre
\textit{lane}, corrispondenti all'ufficio, alla gestione del magazzino
ed all'officina.
Sono state poi modellate alcune \textit{collapsed pool}, che
rappresentano processi di organizzazioni esterne alla ACME: il cliente,
il fornitore, l'istituto di credito, l'ufficio legale e la ditta di
spedizioni.

\subsection{Descrizione del processo ACME}
Il processo dell'azienda ACME da noi descritto viene iniziato dalla
ricezione di un ordine, inviato dalla \textit{collapsed pool} del
cliente.
L'ordine viene catturato da un \textit{message start event}.
Ricevuto l'ordine, esso viene controllato per verificare la presenza di
customizzazioni. Un \textit{exclusive gateway} identifica i due esiti
possibili: o le customizzazioni sono non compatibili, oppure sono
assenti o compatibili.
Nel caso di incompatibilit\`a riscontrate, viene inviato un messaggio al
cliente, contenente il rifiuto dell'ordine. Tale azione viene
implementata tramite un \textit{send task}. Eseguito questo, il processo
termina.
Nel caso le customizzazioni siano invece compatibili, il flusso del
processo passa nella \textit{lane} della gestione magazzini.

Viene verificata la disponibilit\`a dei pezzi che compongono l'ordine:
un \textit{exclusive gateway} divide il flusso del processo nei due
possibili esiti.
Nel caso siano presenti tutti i pezzi, si entra in un
\textit{parallel expanded subprocess}, chiamato Riserva pezzi. Al suo
interno, viene controlla la tipologia di ogni pezzo, in modo da poter
prendere la giusta strada. Un \textit{exclusive gateway} modella questa
divisione. Nel caso il pezzo sia una componente atomica o un accessorio,
viene eseguita una riserva del pezzo dal magazino pi\`u vicino al
cliente; nel caso sia una componente necessaria per l'assemblaggio
ciclo, verr\`a riservato nel magazzino pi\`u vicino all'officina.
Nel caso non vengano trovati tutti i pezzi nei magazzini un
\textit{send task} si occupa di inviare un messaggio al fornitore: tale
messaggio contiene una richiesta di riserva per i pezzi che verrano
ordinati in futuro. Si suppone infatti di avere un accordo con il
fornitore per gestire tale meccanismo.
Il fornitore risponde con una conferma della riserva dei pezzi
desiderati: la risposta viene dunque processata, per verificare che
tutto sia andato a buon fine.

Conclusa questa parte, il flusso del processo arriva ad un
\textit{parallel subprocess}, chiamato ``Riserva pezzi'': per ogni pezzo
infatti viene controllata la tipologia, tramite un
\textit{exclusive gateway}. Se il pezzo \`e una componente atomica
oppure un accessorio allora viene riservata nel magazzino pi\`u vicino
all'utente; altrimenti, cio\`e se \`e un pezzo necessario
all'assemblaggio, viene riservato nel magazzino pi\`u vicino
all'officina.

Uscito dal \textit{subprocess}, il flusso del processo torna nella
\textit{lane} dell'ufficio, dove viene composto il preventivo per il
cliente.
Nel caso il preventivo superi una certa cifra \textit{x} \`e previsto
che venga applicata una percentuale di sconto: un membro dell'ufficio
decide dunque la percentuale di sconto da applicare. Questo meccanismo
viene mappato da un \textit{user task} preceduto da un
\textit{exclusive gateway}. Il preventivo finale viene poi spedito al
cliente e l'ufficio si mette in attesa con un
\textit{event based gateway}.

Vengono dunque mappati tre \textit{intermediate event}:
\begin{itemize}
  \item un \textit{catching message intermediate event}, che attende il
  messaggio di rifiuto del preventivo da parte del cliente;
  \item un \textit{timer intermediate event}, che corrisponde alla
  scadenza di un timer fissato ad un tempo pari a due giorni;
  \item un altro \textit{catching message intermediate event}, che
  attende l'accettazione del preventivo da parte del cliente.
\end{itemize}
I primi due conducono allo stesso \textit{collapsed subprocess},
chiamato ``Annullamento ordine'', che verr\`a descritto successivamente.
Tale \textit{collapsed subprocess} porta poi al termine del processo.

Se l'utente invia invece l'accettazione del preventivo, si verifica che
abbia pagato la caparra, tramite un \textit{collapsed subprocess}
chiamato ``Verifica pagamento'', descritto in seguito. Tale
\textit{collapsed subprocess} comunica con la \textit{pool}
dell'istituto di credito ed ha una scadenza fissata a quindici giorni.
Se questi quindici giorni passano senza che il cliente abbia pagato, il
flusso del processo va nel \textit{collapsed subprocess} ``Annullamento
ordine''; altrimenti, un \textit{exclusive gateway} permette di
controllare se l'importo della caparra \`e maggiore od uguale ad un
decimo del totale. Se \`e cos\`i, allora si procede; altrimenti si va
al \textit{collapsed subprocess} ``Annullamento ordine''.

Si controllano i pezzi necessari: un \textit{exclusive gateway} mappa le
due possibili situazioni, cio\`e se ci sono pezzi da ordinare oppure se
non ci sono. Se ci sono, viene preparata una richiesta da inviare al
fornitore, che contiene l'elenco dei pezzi mancanti. Fatto questo, si
entra in un \textit{collapsed subprocess}, chiamato ``Ricezione pezzi
ordinati'', descritto successivamente.

Fatto tutto questo il flusso del processo passa ad un altro
\textit{parallel collapsed subprocess}, chiamato ``Preparazione ordine e
spedizione'', descritto successivamente. Questo \textit{subprocess} si
occupa della preparazione degli ordini e della loro spedizione,
comunicando con la \textit{lane} dell'officina e con la \textit{pool}
della ditta di spedizioni.

Si torna poi alla \textit{lane} dell'ufficio, con un \textit{send task}
che si occupa di preparare e spedire la conferma della spedizione.
L'ufficio si mette poi in attesa del pagamento, utilizzando il
\textit{collapsed subrocess} di verifica del pagamento visto in
precedenza. Se, come prima, passano quindici giorni, si va verso un
\textit{send task} che si occupa di inviare all'ufficio legale una
notifica di pagamento errato e si termina il processo. Se invece la
verifica del pagamento va a buon fine, si controlla tramite un
\textit{exclusive gateway} se l'importo \`e corretto. Se la risposta
\`e affermativa, l'ordine viene archiviato e il processo finisce; se la
risposta \`e negativa si va al \textit{task} descritto in precedenza che
si occupa di notificare il pagamento errato all'ufficio legale e si
termina il processo.

\subsection{\textit{Collapsed subprocess}}
Durante la descrizione del processo principale dell'azienda ACME si sono
nominati alcuni \textit{collapsed subprocess}: in seguito vengono
spiegati pi\`u nel dettaglio. I \textit{collapsed subprocess} trattati
sono i seguenti:
\begin{itemize}
  \item ``Annullamento ordine'';
  \item ``Ricezione pezzi ordinati'';
  \item ``Preparazione ordine e spedizione'';
  \item ``Verifica pagamento''.
\end{itemize}
\subsubsection{Annullamento ordine}
Il \textit{suprocess} inizia dalla \textit{lane} dell'ufficio, ma passa
subito a quella della gestione magazzini: qui vengono rilasciati i pezzi
riservati con un \textit{parallel task}. Il flusso del processo torna
poi all'ufficio, dove viene preparato e spedito un messaggio di
annullamento dell'ordine alla \textit{pool} del cliente.
\subsubsection{Ricezione pezzi ordinati}
Il \textit{subprocess} parte dalla \textit{lane} della gestione
magazzini. Arrivati i pezzi, vengono sistemati e il \textit{subprocess}
termina.
\subsubsection{Preparazione ordine e spedizione}
Il \textit{parallel subprocess} parte dalla \textit{lane} della gestione
magazzini, con un \textit{task} che si occupa del controllo della
tipologia del pezzo. I casi sono due: o il pezzo \`e una componente
atomica (o un accessorio) oppure una componente da assemblare.

Se si tratta una componente atomica o un accessorio, viene preparato un
pacco dal magazzino pi\`u vicino al cliente, poi viene preparata una
richiesta di spedizione con un \textit{send task}; dopo la ricezione
della conferma della spedizione, si conclude il \textit{subprocess}.

Se invece \`e una componente da assemblare, viene inviato alla
\textit{lane} dell'officina. Qui viene assemblato il ciclo, poi tramite
un \textit{send task} si effettua la richiesta di spedizione alla ditta
di spedizione. Ricevuta la conferma, il \textit{subprocess} termina.
\subsubsection{Verifica pagamento}
Il \textit{subprocess} parte dalla \textit{lane} dell'ufficio, dove un
\textit{send task} si occupa di preparare e spedire la richiesta di
verifica di pagamento alla \textit{pool} dell'istituto di credito. Viene
attesa la risposta e se ne controlla il contenuto: se il pagamento \`e
stato effettuato, il \textit{subprocess} termina; altrimenti si
attendono ventiquattro ore e si effettua di nuovo la richiesta.

