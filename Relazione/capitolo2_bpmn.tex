\section{Diagramma BPMN}
Il diagramma BPMN \`e stato realizzato utilizzando il software di
modellazione online \textit{Signavio}.
Sono state modellate diverse \textit{pool}: la principale, in chiaro,
\`e quella dell'azienda ACME, suddivisa al proprio interno in tre
\textit{lane}, corrispondenti all'ufficio, alla gestione del magazzino
ed all'officina.
Sono state poi modellate alcune \textit{collapsed pool}, che
rappresentano processi di organizzazioni esterne alla ACME: il cliente,
il fornitore, l'istituto di credito, l'ufficio legale e la ditta di
spedizioni.

\subsection{Descrizione del processo ACME}
Il processo dell'azienda ACME da noi descritto viene iniziato dalla
ricezione di un ordine, inviato dalla \textit{collapsed pool} del
cliente.
L'ordine viene catturato da un \textit{message start event}.
Ricevuto l'ordine, esso viene controllato per verificare la presenza di
customizzazioni. Un \textit{exclusive gateway} identifica i due esiti
possibili: o le customizzazioni sono non compatibili, oppure sono
assenti o compatibili.
Nel caso di incompatibilit\`a riscontrate, viene inviato un messaggio al
cliente, contenente il rifiuto dell'ordine. Tale azione viene
implementata tramite un \textit{send task}. Eseguito questo, il processo
termina.
Nel caso le customizzazioni siano invece compatibili, il flusso del
processo passa nella \textit{lane} della Gestione Magazzini.

Viene verificata la disponibilit\`a dei pezzi che compongono l'ordine:
un \textit{exclusive gateway} divide il flusso del processo nei due
possibili esiti.
Nel caso siano presenti tutti i pezzi, si entra in un
\textit{parallel expanded subprocess}, chiamato Riserva pezzi. Al suo
interno, viene controlla la tipologia di ogni pezzo, in modo da poter
prendere la giusta strada. Un \textit{exclusive gateway} modella questa
divisione. Nel caso il pezzo sia una componente atomica o un accessorio,
viene eseguita una riserva del pezzo dal magazino pi\`u vicino al
cliente; nel caso sia una componente necessaria per l'assemblaggio
ciclo, verr\`a riservato nel magazzino pi\`u vicino all'officina.
Nel caso non vengano trovati tutti i pezzi nei magazzini un
\textit{send task} si occupa di inviare un messaggio al fornitore: tale
messaggio contiene una richiesta di riserva per i pezzi che verrano
ordinati in futuro. Si suppone infatti di avere un accordo con il
fornitore per gestire tale meccanismo.
Il fornitore risponde con una conferma della riserva dei pezzi
desiderati: la risposta viene dunque processata, per verificare che
tutto sia andato a buon fine.

Conclusa questa parte, il flusso del processo arriva ad un
\textit{parallel subprocess}, chiamato ``Riserva pezzi'': per ogni pezzo
infatti viene controllata la tipologia, tramite un
\textit{exclusive gateway}. Se il pezzo \`e una componente atomica
oppure un accessorio allora viene riservata nel magazzino pi\`u vicino
all'utente; altrimenti, cio\`e se \`e un pezzo necessario
all'assemblaggio, viene riservato nel magazzino pi\`u vicino
all'officina.

Uscito dal \textit{subprocess}, il flusso del processo torna nella
\textit{lane} dell'Ufficio, dove viene composto il preventivo per il
cliente.
Nel caso il preventivo superi una certa cifra \textit{x} \`e previsto
che venga applicata una percentuale di sconto: un membro dell'ufficio
decide dunque la percentuale di sconto da applicare. Questo meccanismo
viene mappato da un \textit{user task} preceduto da un
\textit{exclusive gateway}. Il preventivo finale viene poi spedito al
cliente e l'ufficio si mette in attesa con un
\textit{event based gateway}.

Vengono dunque mappati tre \textit{intermediate event}:
\begin{itemize}
  \item un \textit{catching message intermediate event}, che attende il
  messaggio di rifiuto del preventivo da parte del cliente;
  \item un \textit{timer intermediate event}, che corrisponde alla
  scadenza di un timer fissato ad un tempo pari a due giorni;
  \item un altro \textit{catching message intermediate event}, che
  attende l'accettazione del preventivo da parte del cliente.
\end{itemize}
I primi due conducono allo stesso \textit{collapsed subprocess},
chiamato ``Annullamento ordine'', che verr\`a descritto successivamente.
Tale \textit{collapsed subprocess} porta poi al termine del processo.

Se l'utente invia invece l'accettazione del preventivo, si verifica che
abbia pagato la caparra, tramite un \textit{collapsed subprocess}
chiamato ``Verifica pagamento'', descritto in seguito. Tale
\textit{collapsed subprocess} comunica con la \textit{lane}
dell'Istituto di credito ed ha una scadenza fissata a quindici giorni.
Se questi quindici giorni passano senza che il cliente abbia pagato, il
flusso del processo va nel \textit{collapsed subprocess} ``Annullamento
ordine''; altrimenti, un \textit{exclusive gateway} permette di
controllare se l'importo della caparra \`e maggiore od uguale ad un
decimo del totale. Se \`e cos\`i, allora si procede; altrimenti si va
al \textit{collapsed subprocess} ``Annullamento ordine''.
===================
Ricevuto l'esito positivo dall'istituto di credito, un
\textit{exclusive gateway} consente di smistare gli ordini nelle due
categorie gi\`a incontrate: cicli e componenti. Il processo viene dunque
spostato nella \textit{lane} della gestione del magazzino.

Nel caso l'ordine contenga un ciclo, si controlla la presenza delle
componenti nel magazzino primario, una per volta. Un
\textit{exclusive gateway} mappa le due possibilit\`a: o la componente
viene trovata, oppure non viene trovata.
Se non viene trovata si cerca all'interno dei magazzini secondari:
anche in questo caso un \textit{exclusive gateway} mappa i due risultati
possibili. Se viene trovata la si spedisce al magazzino primario,
altrimenti la si ordina. Viene dunque mandato un ordine alla
\textit{collapsed pool} del fornitore, il quale provvede a spedire la
componente al magazzino primario. Quando tutte le componenti sono
arrivate al magazzino primario, si procede alla fase di assemblaggio,
nella \textit{lane} dell'officina. L'officina assembla il ciclo, prepara
la spedizione e si affida alla ditta di spedizioni.

Nel caso l'ordine contenga delle componenti si verifica tramite un
\textit{exclusive gateway} che siano gi\`a state riservate; nel caso non
sia cos\`i, le componenti vengono ordinate al fornitore, con la
procedura vista in precedenza. Quando tutte le componenti sono presenti,
anche se sparse in diversi magazzini, vengono preparate alla consegna e
date alla ditta di spedizioni.

Quando all'officina arriva la notifica di consegna avvenuta, catturata
da un \textit{catching message boundary event}, il processo torna
all'ufficio.

L'ufficio riparte dall'attesa del saldo, affidandosi ad un
\textit{event based gateway}. Gli eventi che l'ufficio attende sono due:
un \textit{catching message intermediate event}, che rappresenta
l'arrivo del saldo dal cliente, oppure un
\textit{timer intermediate event}, che rappresenta la fine di un'attesa
fissata ad un tempo \textit{y}. Se scade il timer, viene inviata una
notifica all'ufficio legale, per poi terminare il processo.
Se invece arriva il saldo dal cliente, lo si invia all'istituto di
credito, aspettando l'esito. Un \textit{exclusive gateway} basato
sull'esito della verifica del bonifico indirizza il processo a due
possibili terminazioni: se l'esito \`e negativo, si invia una notifica
all'ufficio legale e poi si termina; se l'esito \`e positivo, il
processo termina senza ulteriori passaggi.
