\section{Diagramma BPMN}
Il diagramma BPMN \`e stato realizzato utilizzando il software di
modellazione online \textit{Signavio}.
Sono state modellate diverse \textit{pool}: la principale, in chiaro,
\`e quella dell'azienda ACME, suddivisa al proprio interno in tre
\textit{lane}, corrispondenti all'ufficio, alla gestione del magazzino
ed all'officina.
Sono state poi modellate alcune \textit{collapsed pool}, che
rappresentano processi di organizzazioni esterne alla ACME: il cliente,
il fornitore, l'istituto di credito, l'ufficio legale e la ditta di
spedizioni.

\subsection{Descrizione del processo ACME}
Il processo dell'azienda ACME da noi descritto viene iniziato dalla
ricezione di un ordine, inviato dalla \textit{collapsed pool} del
cliente.
L'ordine viene catturato da un \textit{message start event}.
Ricevuto l'ordine, esso viene controllato per verificare la presenza di customizzazioni. Un \textit{exclusive gateway} identifica i due esiti possibili: o le customizzazioni sono compatibili, oppure sono assenti o compatibili.
Nel caso di incompatibilit\`a riscontrate, viene inviato un messaggio al cliente, contenente il rifiuto dell'ordine. Tale azione viene implementata tramite un \textit{send task}. Eseguito questo, il processo termina.
Nel caso le customizzazioni siano invece compatibili, il flusso del processo passa nella \textit{lane} della Gestione Magazzini.

Come prima cosa viene verificata la disponibilit\`a dei pezzi che compongono l'ordine: un \textit{exclusive gateway} divide il flusso del processo nei due possibili esiti.
Nel caso siano presenti tutti i pezzi, si entra in un \textit{parallel expanded subprocess}, chiamato Riserva pezzi. Al suo interno, viene controlla la tipologia di ogni pezzo, in modo da poter prendere la giusta strada. Un \textit{exclusive gateway} modella questa divisione. Nel caso il pezzo sia una componente atomica o un accessorio, viene eseguita una riserva del pezzo dal magazino pi\`u vicino al cliente; nel caso sia una componente necessaria per l'assemblaggio ciclo, verr\`a riservato nel magazzino pi\`u vicino all'officina.
Nel caso non vengano trovati tutti i pezzi nei magazzini .... DA COMPLETARE

Conclusa questa parte, il flusso del processo torna nella \textit{lane} dell'Ufficio, dove viene composto il preventivo per il cliente.
Nel caso il preventivo superi una certa cifra \textit{x}, un membro dell'ufficio decide l'applicazione di uno sconto. Questo meccanismo viene mappato da un \textit{user task} seguito da un \textit{exclusive gateway}. Il preventivo finale viene poi spedito al cliente e l'ufficio si mette in attesa con un \textit{event based gateway}.

Gli \textit{intermediate event} mappati
sono tre: o un \textit{catching message intermediate event}, che
corrisponde alla risposta del cliente, oppure un
\textit{timer intermediate event}, che corrisponde alla scadenza di un
timer fissato ad un tempo \textit{y}.
===========

Nel caso non vengano accettate come compatibili, viene eseguito un
\textit{message end event}, che corrisponde all'invio al cliente di una
notifica di annullamento dell'ordine.
In caso le customizzazioni vengano accettate, si passa ad un altro
\textit{exclusive gateway}, che smista gli ordini nelle due categorie
previste: ordini che contengono cicli e ordini che contengono
componenti.
Questo passaggio serve a determinare il costo di spedizione che verr\`a
addebitato al cliente.

In caso l'ordine contenga un ciclo, il prezzo di spedizione viene
calcolato prendendo come parametri l'indirizzo del cliente e quello del
magazzino primario: questo perch\'e i cicli vengono assemblati
all'interno del magazzino primario.

Nel caso l'ordine contenga delle componenti, si effettuer\`a una ricerca
\linebreak all'interno di tutti i magazzini (utilizzando la lane della gestione
magazzini), per ogni componente. Un \textit{exclusive gateway} mappa i
due possibili esiti: se la componente viene trovata in uno o pi\`u
magazzini, il costo di spedizione si baser\`a sull'indirizzo del cliente
e quello del magazzino pi\`u vicino al cliente che contiene la
componente; nel caso la componente non venga trovata, il costo di
spedizione viene calcolato in base all'indirizzo del cliente e a quello
del magazzino primario, dato che il fornitore spedisce solamente a
quello.
Le componenti vengono riservate, qualora si trovino in un magazzino, ma
non vengono ordinate al fornitore nel caso non ci siano.

Se scade il timer, si comunica la magazzino di rilasciarele componenti e
si arriva ad un \textit{message end event}, che corrisponde all'invio
della notifica di annullamento dell'ordine al cliente. Nel caso il
cliente invii una risposta, un \textit{exclusive gateway} mappa le due
possibili risposte: accettazione o rifiuto del preventivo. Nel caso di
rifiuto, si attua la stessa procedura descritta nel caso della scadenza
del timer.

L'ufficio si mette poi in attesa dell'acconto dal cliente, con un
\textit{timer boundary event}, che avvia la procedura di annullamento
dell'ordine vista in precedenza, dopo la scadenza di un tempo
\textit{y}. L'acconto viene mandato dal cliente sotto forma di un
bonifico: se tale bonifico \`e inferiore ad un decimo della somma
indicata dal preventivo finale, viene avviata la procedura di
annullamento vista in precedenza; altrimenti il bonifico viene mandato
alla \textit{collapsed pool} dell'istituto di credito.

In base all'esito sulla verifica della solvibilit\`a del bonifico
comunicato \linebreak dall'istituto di credito, l'ufficio decide se annullare
l'ordine seguendo la procedura indicata in precedenza oppure procedere.
Questo meccanismo viene implementato da un \textit{exclusive gateway}.

Ricevuto l'esito positivo dall'istituto di credito, un
\textit{exclusive gateway} consente di smistare gli ordini nelle due
categorie gi\`a incontrate: cicli e componenti. Il processo viene dunque
spostato nella \textit{lane} della gestione del magazzino.

Nel caso l'ordine contenga un ciclo, si controlla la presenza delle
componenti nel magazzino primario, una per volta. Un
\textit{exclusive gateway} mappa le due possibilit\`a: o la componente
viene trovata, oppure non viene trovata.
Se non viene trovata si cerca all'interno dei magazzini secondari:
anche in questo caso un \textit{exclusive gateway} mappa i due risultati
possibili. Se viene trovata la si spedisce al magazzino primario,
altrimenti la si ordina. Viene dunque mandato un ordine alla
\textit{collapsed pool} del fornitore, il quale provvede a spedire la
componente al magazzino primario. Quando tutte le componenti sono
arrivate al magazzino primario, si procede alla fase di assemblaggio,
nella \textit{lane} dell'officina. L'officina assembla il ciclo, prepara
la spedizione e si affida alla ditta di spedizioni.

Nel caso l'ordine contenga delle componenti si verifica tramite un
\textit{exclusive gateway} che siano gi\`a state riservate; nel caso non
sia cos\`i, le componenti vengono ordinate al fornitore, con la
procedura vista in precedenza. Quando tutte le componenti sono presenti,
anche se sparse in diversi magazzini, vengono preparate alla consegna e
date alla ditta di spedizioni.

Quando all'officina arriva la notifica di consegna avvenuta, catturata
da un \textit{catching message boundary event}, il processo torna
all'ufficio.

L'ufficio riparte dall'attesa del saldo, affidandosi ad un
\textit{event based gateway}. Gli eventi che l'ufficio attende sono due:
un \textit{catching message intermediate event}, che rappresenta
l'arrivo del saldo dal cliente, oppure un
\textit{timer intermediate event}, che rappresenta la fine di un'attesa
fissata ad un tempo \textit{y}. Se scade il timer, viene inviata una
notifica all'ufficio legale, per poi terminare il processo.
Se invece arriva il saldo dal cliente, lo si invia all'istituto di
credito, aspettando l'esito. Un \textit{exclusive gateway} basato
sull'esito della verifica del bonifico indirizza il processo a due
possibili terminazioni: se l'esito \`e negativo, si invia una notifica
all'ufficio legale e poi si termina; se l'esito \`e positivo, il
processo termina senza ulteriori passaggi.
