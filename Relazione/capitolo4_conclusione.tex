\section{Conclusioni}

La soluzione adottata per implementare il sistema di vendite e gestione
dell'ordine dell'azienda fittizia ACME ha avuto, durante tutte le fasi
della sua implementazione, come unico obiettivo quello di simulare i
meccanismi di un sistema reale, ovviamente mantenendo le linee guida
indicate dalle specifiche.

Simulare un sistema non vuol dire per\`o trascurare gli aspetti pi\`u
importanti implementati nelle soluzioni che ci circondano: semplicemente
si \`e andati a sviluppare meno componenti software elaborate, come per
esempio un vero e proprio sistema di e-commerce, per concentrarsi sulla
spiegazione di meccanismi fondamentali per soluzioni del genere, come la
comunicazione con il cliente, la gestione del sistema tramite software
come Camunda, l'implementazione di microservizi Jolie, la progettazione
tramite schemi BPMN.

In altre parole, la simulazione di certe parti con meccanismi
semplificati rispetto a quelli effettivi \`e stata fatta per avere
maggiori risorse impiegate nella comprensione e nello svolgimento delle
meccaniche fondamentali del progetto: semplificando la parte estetica,
\`e stato dunque possibile sviluppare approfonditamente quella logica.
